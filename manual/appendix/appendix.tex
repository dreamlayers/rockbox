% $Id$ %
\appendix

% $Id$ %
\chapter{File formats}
\section{\label{ref:Supportedfileformats}Supported file formats}
\begin{rbtabular}{\textwidth}{cl>{\raggedright}p{7em}X}%
{\textbf{Icon} & \textbf{File Type} & \textbf{Extension} 
  & \textbf{Action when selected}}{}{}
\includegraphics[width=0.37cm]{appendix/images/icon-directory.png} 
  & Directory & \emph{none} & Enter the directory \tabularnewline
\opt{recorder,recorderv2fm,ondiofm,ondiosp}{
  \includegraphics[width=0.37cm]{appendix/images/icon-rolo.png} 
  & Rockbox firmware & \fname{.ajz} & Load the new firmware with ROLO \tabularnewline
}
\opt{swcodec}{
  \includegraphics[width=0.37cm]{appendix/images/icon-audio-file.png} 
  & Audio file & \emph{various}\newline%
  (see \ref{ref:Supportedaudioformats})%
  % do NOT use \reference{} here as that will break the table.
  & Start playing the file and show the WPS\tabularnewline
}
  & Bookmark & \fname{.bmark} & Display all bookmarks for an audio file\tabularnewline
\opt{lcd_bitmap}{
  & Game of Life & \fname{.cells} & Show the configuration with the
     ``Rocklife'' plugin\tabularnewline
}
\includegraphics[width=0.37cm]{appendix/images/icon-config.png} 
  & Configuration File & \fname{.cfg} & Load the settings file\tabularnewline
\includegraphics[width=0.37cm]{appendix/images/icon-chip8.png} 
  & Chip8 game & \fname{.ch8} & Play the Chip8 game \tabularnewline
\opt{lcd_color}{
  & Colours & \fname{.colours} & Open the colours file for editing.
    See \reference{ref:ChangingFiletypeColours}.\tabularnewline
}
\includegraphics[width=0.37cm]{appendix/images/icon-cuesheet.png} 
  & Cuesheet & \fname{.cue} & View the cuesheet file \tabularnewline
\opt{radio}{
  & FM Presets & \fname{.fmr} & Load the FM Presets (previous are discarded)\tabularnewline
}
\includegraphics[width=0.37cm]{appendix/images/icon-font.png} 
  & Font & \fname{.fnt} & Change the user interface font to this one\tabularnewline
\opt{gigabeat}{
  \includegraphics[width=0.37cm]{appendix/images/icon-rolo.png} 
  & Rockbox firmware & \fname{.gigabeat} & Load the new firmware with ROLO \tabularnewline
}
\opt{iaudio}{
  \includegraphics[width=0.37cm]{appendix/images/icon-rolo.png} 
  & Rockbox firmware & \fname{.iaudio} & Load the new firmware with ROLO \tabularnewline
}
\opt{ipod}{
  \includegraphics[width=0.37cm]{appendix/images/icon-rolo.png} 
  & Rockbox firmware & \fname{.ipod} & Load the new firmware with ROLO \tabularnewline
}
\opt{iriverh100,iriverh300}{
  \includegraphics[width=0.37cm]{appendix/images/icon-rolo.png} 
  & Rockbox firmware & \fname{.iriver} & Load the new firmware with ROLO \tabularnewline
}
\includegraphics[width=0.37cm]{appendix/images/icon-image-file.png} 
  & Image & \fname{.jpg} & View the JPEG image \tabularnewline
  & Link & \fname{.link} & Display list of target files and directories;
    selecting one jumps to the target. See \reference{ref:Shortcutsplugin}.\tabularnewline
\includegraphics[width=0.37cm]{appendix/images/icon-lang.png} 
  & Language File & \fname{.lng} & Load the language file \tabularnewline
\includegraphics[width=0.37cm]{appendix/images/icon-playlist.png}
  & Playlist & \fname{.m3u}, \fname{.m3u8} & Load the playlist and start playing 
    the first file \tabularnewline
\opt{iriverh10,iriverh10_5gb,sansa,mrobe100,vibe500,samsungyh}{
  \includegraphics[width=0.37cm]{appendix/images/icon-rolo.png} 
  & Rockbox firmware & \fname{.mi4} & Load the new firmware with ROLO \tabularnewline
}
\opt{player}{
  \includegraphics[width=0.37cm]{appendix/images/icon-rolo.png} 
  & Rockbox firmware & \fname{.mod} & Load the new firmware with ROLO \tabularnewline
}
\opt{masd,masf}{
  \includegraphics[width=0.37cm]{appendix/images/icon-audio-file.png} 
  & Audio file & \fname{.mp2}, \fname{.mp3} & Start playing the file and show the WPS\tabularnewline
}
\opt{swcodec}{
 \includegraphics[width=0.37cm]{appendix/images/icon-movie-file.png}
 & Video & \fname{.mpg}, \fname{.mpeg}, \fname{.mpv}, \fname{.m2v} & Play the MPEG1/2 video \tabularnewline
}
\includegraphics[width=0.37cm]{appendix/images/icon-rock.png} 
  & Plugin & \fname{.rock} & Start the plugin\tabularnewline
\opt{masf}{\opt{lcd_bitmap}{
  \includegraphics[width=0.37cm]{appendix/images/icon-movie-file.png} 
    & Rockbox Video & \fname{.rvf} & View the movie (Rockbox format)\tabularnewline}
}
\opt{sansaAMS}{
  \includegraphics[width=0.37cm]{appendix/images/icon-rolo.png} 
  & Rockbox firmware & \fname{.sansa} & Load the new firmware with ROLO \tabularnewline
}
\includegraphics[width=0.37cm]{appendix/images/icon-text.png} 
  & Text File & \fname{.txt} & Display the text file using the text viewer plugin\tabularnewline
\opt{archos}{
  \includegraphics[width=0.37cm]{appendix/images/icon-ucl.png} 
    & Flash Image & \fname{.ucl} & Flash the Rockbox image into the ROM \tabularnewline
  }
  & Voice file & \fname{.voice} & Allow Rockbox to speak menus\tabularnewline
\opt{masf}{
  \includegraphics[width=0.37cm]{appendix/images/icon-wav-file.png} 
    & Wave Audio File & \fname{.wav} & Play the WAV file \tabularnewline%
}
\includegraphics[width=0.37cm]{appendix/images/icon-wps.png} 
  & While Playing Screen & \fname{.wps} & Load the new WPS display configuration\tabularnewline
\end{rbtabular}

\opt{swcodec}{
  \chapter{Audio and metadata formats}
  \section{\label{ref:Supportedaudioformats}Supported audio formats}
  \subsection{Lossy Codecs}
  \begin{rbtabular}{\textwidth}{l>{\raggedright}p{6em}X}%
  {\textbf{Format} & \textbf{Extension} & \textbf{Notes}}{}{}
    ATSC A/52 (AC3)
        & \fname{.a52}, \fname{.ac3}, \fname{.rm}, \fname{.ra}, \fname{.rmvb}
        & Supports downmixing for playback of 5.1 streams in stereo\\
    ADX
        & \fname{.adx} 
        & Encrypted ADX is not supported.\\
    Advanced Audio Coding
        & \fname{.m4a}, \fname{.m4b}, \fname{.mp4}, \fname{.rm}, \fname{.ra}, \fname{.rmvb}
        \nopt{clipv1,c200v2}{
            & Supports AAC-LC, -HEv1, and -HEv2 profiles\\}
        \opt{clipv1,c200v2}{ % low memory targets (CODEC_SIZE <= 512 KB)
            & Supports AAC-LC profile\\}
    MPEG audio
        & \fname{.mpa}, \fname{.mp1}, \fname{.mp2}, \fname{.mp3} 
        & MPEG 1/2/2.5 Layer 1/2/3\\
    Musepack
        & \fname{.mpc} 
        & Supports SV7 and SV8 in mono/stereo \\
    OGG/Vorbis
        & \fname{.ogg}, \fname{.oga} 
        & Playback of some old ``floor 0'' files may fail on low memory targets.
          Files with album art larger than available RAM will be skipped. 
          Chained Ogg files are not supported.\\
    Sony Audio
        & \fname{.oma}, \fname{.aa3}, \fname{.rm}, \fname{.ra}, \fname{.rmvb}
        & Supports ATRAC3\\
    RealAudio
        & \fname{.rm}, \fname{.ra}, \fname{.rmvb}
        & Supports RealAudio G2 (Cook)\\
    Speex
        & \fname{.spx} 
        & \\
    Dialogic telephony type
        & \fname{.vox} 
        & \\
    Windows Media Audio Standard
        & \fname{.wma}, \fname{.wmv}, \fname{.asf} 
        & \\
    Windows Media Audio Professional
        & \fname{.wma}, \fname{.wmv}, \fname{.asf} 
        & \\
  \end{rbtabular}
  
  \note{AAC-HE profiles might not play in realtime on all devices due to CPU 
  performance requirements.}

  \subsection{Lossless Codecs}
  \begin{rbtabular}{\textwidth}{lp{6em}X}%
  {\textbf{Format} & \textbf{Extension} & \textbf{Notes}}{}{}
    Audio Interchange File Format
        & \fname{.aif}, \fname{.aiff} 
        & Linear PCM 8/16/24/32 bit, IEEE float 32/64 bit, ITU-T G.711 a-law/$\mu$-law,
          QuickTime IMA ADPCM\\
    Monkey's Audio
        & \fname{.ape}, \fname{.mac} 
        & 
        \opt{gigabeatf,iriverh100,iriverh300,iaudiox5,iaudiom5,iaudiom3,ipodnano2g,clipv1}{
            -c1000 to -c3000 files decode fast enough to be useful.}
        \opt{gigabeats}{}
        \opt{ipod,iriverh10,iriverh10_5gb,mrobe100,sansa,vibe500,samsungyh}{
            \nopt{ipodnano2g}{Only -c1000 files decode fast enough to be useful.}}
            \\
    Sun Audio
        & \fname{.au}, \fname{.snd} 
        & Linear PCM 8/16/24/32 bit, IEEE float 32/64 bit, ITU-T G.711 a-law/$\mu$-law\\
    Free Lossless Audio
        & \fname{.flac} 
        & Supports multichannel playback including downmixing to stereo.\\
    Apple Lossless
        & \fname{.m4a}, \fname{.mp4} 
        & \\
    Shorten
        & \fname{.shn} 
        & Seeking not supported.\\
    True Audio
        & \fname{.tta} 
        & \\
    Wave64
        & \fname{.w64} 
        & Supports same formats as Waveform audio format.\\
    Waveform audio format
        & \fname{.wav} 
        & Linear PCM 8/16/24/32 bit, IEEE float 32/64 bit, ITU-T G.711 a-law/$\mu$-law,
          Microsoft ADPCM, Intel DVI ADPCM (IMA ADPCM) 2/3/4/5 bit, Dialogic OKI ADPCM,
          YAMAHA ADPCM, Adobe SWF ADPCM\\
    Wavpack
        & \fname{.wv} 
        & \\
  \end{rbtabular}
  
  \note{Free Lossless Audio multichannel tracks may not play in realtime on all devices due to CPU 
  performance requirements.}

  \subsection{Other Codecs}
  \begin{rbtabular}{\textwidth}{l>{\raggedright}p{6em}X}%
  {\textbf{Format} & \textbf{Extension} & \textbf{Notes}}{}{}
    Atari Sound Format
        & \fname{.cmc}, \fname{.cm3}, \fname{.cmr}, \fname{.cms}, \fname{.dmc}, 
          \fname{.dlt}, \fname{.mpt}, \fname{.mpd} 
        & \\
    Synthetic music Mobile Application Format
        & \fname{.mmf} 
        & PCM/ADPCM only \\
    Game Boy Sound Format
        & \fname{.gbs}
        & Progress bar and seek use subtracks instead of seconds.\\
    AY Sound Chip Music
        & \fname{.ay}
        & Progress bar and seek use subtracks instead of seconds for
          multitrack files.\\
    Hudson Entertainment System Sound Format
        & \fname{.hes}
        & Progress bar and seek use subtracks instead of seconds.\\
    \nopt{clipv1,c200v2}{
    MSX Konami Sound System
        & \fname{.kss}
        & Progress bar and seek use subtracks instead of seconds.\\}
    SMS/GG/CV Sound Format
        & \fname{.sgc}
        & Supports Sega Master System and Game Gear Sound Format. 
          Progress bar and seek use subtracks instead of seconds.\\
    Video Game Music Format
        & \fname{.vgm}
        & \\
    Gzipped Video Game Music Format
        & \fname{.vgz}
        & \\
    MOD
        & \fname{.mod} 
        & \\
    NES Sound Format
        & \fname{.nsf}, \fname{.nsfe} 
        & Progress bar and seek use subtracks instead of seconds.\\
    Atari SAP
        & \fname{.sap} 
        & \\
    Sound Interface Device
        & \fname{.sid} 
        & Progress bar and seek use subtracks instead of seconds.\\
    SPC700
        & \fname{.spc} 
        & \\
  \end{rbtabular}
  
  \note{NSF and VGM might not play in realtime on all devices due to CPU 
  performance requirements.}
  
  \subsection{Codec featureset}
  \begin{rbtabular}{.95\textwidth}{lXXX}%
  {\textbf{Format} & \textbf{Seek} & \textbf{Resume} & \textbf{Gapless}}{}{}
    ATSC A/52 (AC3)                             & x & x &   \\
    ADX                                         & x &   &   \\
    Advanced Audio Coding                       & x & x & x \\
    MPEG audio                                  & x & x & x \\
    Musepack                                    & x & x & x \\
    OGG/Vorbis                                  & x & x & x \\
    Sony Audio                                  & x & x &   \\
    RealAudio                                   & x & x &   \\
    Dialogic telephony type                     & x & x &   \\
    Windows Media Audio Standard                & x & x &   \\
    Windows Media Audio Professional            & x & x &   \\
    Audio Interchange File Format               & x & x & x \\
    Monkey's Audio                              & x & x & x \\
    Sun Audio                                   & x & x & x \\
    Free Lossless Audio                         & x & x & x \\
    Apple Lossless                              & x & x & x \\
    Shorten                                     &   &   & x \\
    True Audio                                  & x & x & x \\
    Wave64                                      & x & x & x \\
    Waveform audio format                       & x & x & x \\
    Wavpack                                     & x & x & x \\
    Atari Sound Format                          & x &   &   \\
    Synthetic music Mobile Application Format   & x & x &   \\
    Game Boy Sound Format                       & x &   &   \\
    AY Sound Chip Music                         & x &   &   \\
    Hudson Entertainment System Sound Format    & x &   &   \\
    MSX Konami Sound System                     & x &   &   \\
    SMS/GG/CV Sound Format                      & x &   &   \\
    Video Game Music Format                     & x & x &   \\
    Gzipped Video Game Music Format             & x & x &   \\
    MOD                                         & x &   &   \\
    NES Sound Format                            & x &   &   \\
    Atari SAP                                   & x &   &   \\
    Sound Interface Device                      & x &   &   \\
    SPC700                                      & x &   &   \\
  \end{rbtabular}
  
  \note{The seek implementations of NES Sound Format, Sound Interface Device,
  Game Boy Sound Format, AY Sound Chip Music, Hudson Entertainment System Sound,
  Format, MSX Konami Sound System and SMS/GG/CV Sound Format use subtracks
  instead of seconds, whereas each subtrack equals a second.}
  
  \section{\label{ref:SupportedMetadata}Supported metadata tags}
    Rockbox supports different metadata formats. In general those tag formats
    are ID3 (v1.0, v1.1, v2.2, v2.3 and v2.4), APE (v1 and v2), Vorbis, MP4 and 
    ASF. Few codecs use codec specific tags, several codecs do not use any tags 
    yet. The following table gives an overview about what tag types rockbox 
    supports for which audio file extension.
    
    \note{There is always only \emph{one} tag type supported for each file
    extension.}
    
    \begin{rbtabular}{\textwidth}{lX}%
    {\textbf{Tag type} & \textbf{File extension}}{}{}
      ID3               & \fname{.mp1}, \fname{.mpa}, \fname{.mp2}, \fname{.mp3},
                          \fname{.rm}, \fname{.ra}, \fname{.rmvb}, \fname{.tta} \\
      APE               & \fname{.mpc}, \fname{.ape}, \fname{.mac}, \fname{.wv} \\
      Vorbis            & \fname{.ogg}, \fname{.oga}, \fname{.spx}, \fname{.flac} \\
      MP4               & \fname{.m4a}, \fname{.m4b}, \fname{.mp4} \\
      ASF               & \fname{.wma}, \fname{.wmv}, \fname{.asf} \\
      Codec specific    & \fname{.mmf}, \fname{.mod}, \fname{.nsf}, \fname{.nsfe},
                          \fname{.sap}, \fname{.sid}, \fname{.spc}, \fname{.gbs},
                          \fname{.ay}, \fname{.kss}, \fname{.sgc}, \fname{.vgm} \\
      None              & \fname{.a52}, \fname{.ac3}, \fname{.adx}, \fname{.oma},
                          \fname{.aa3}, \fname{.aif}, \fname{.aiff}, \fname{.au},
                          \fname{.snd}, \fname{.shn}, \fname{.vox}, \fname{.w64},
                          \fname{.wav}, \fname{.cmc}, \fname{.cm3}, \fname{.cmr},
                          \fname{.cms}, \fname{.dmc}, \fname{.dlt}, \fname{.mpt},
                          \fname{.mpd}, \fname{.hes}, \fname{.vgz} \\
    \end{rbtabular}
    
    \subsection{Featureset for generic metadata tags}
    \label{ref:featureset_for_generic_metadata_tags}
    \begin{rbtabular}{0.90\textwidth}{lXXXXX}%
    {\textbf{Feature} & \textbf{ID3} & \textbf{APE} & \textbf{Vorbis} & 
     \textbf{MP4} & \textbf{ASF}}{}{}
     Embedded albumart \fname{.bmp}     &   &   &   &   &   \\
     Embedded albumart \fname{.jpg}     & x & x &   & x & x \\
     Embedded albumart \fname{.png}     &   &   &   &   &   \\
     Embedded cuesheet                  & x &   & x &   &   \\
     Replaygain information             & x & x & x & x & x \\
     Title (string)                     & x & x & x & x & x \\
     Artist (string)                    & x & x & x & x & x \\
     Album (string)                     & x & x & x & x & x \\
     Genre (string)                     & x & x & x & x & x \\
     Disc (string or number)            & x & x & x & x &   \\
     Track (string or number)           & x & x & x & x & x \\
     Year (string or number)            & x & x & x & x & x \\
     Composer (string)                  &   & x & x & x & x \\
     Comment (string)                   & x & x & x & x & x \\
     Albumartist (string)               & x & x & x & x & x \\
     Grouping (string)                  &   & x & x & x &   \\
    \end{rbtabular}
    
    \note{Embedded album art for ASF is limited to pictures of maximum 64 KB size.}
    
    \subsection{Featureset for codec specific metadata}
    \begin{rbtabular}{\textwidth}{lX}%
    {\textbf{Feature} & \textbf{Codec specific metadata (file extension)}}{}{}
     Embedded \fname{.bmp}  & None \\
     Embedded \fname{.jpg}  & None \\
     Embedded \fname{.png}  & None \\
     Replaygain             & \fname{.mpc}\\
     Title                  & \fname{.tta}, \fname{.spc}, \fname{.mmf}, \fname{.sid}, 
                              \fname{.rm}, \fname{.ra}, \fname{.rmvb}, \fname{.nsf}, 
                              \fname{.nsfe}, \fname{.mod}, \fname{.sap}, \fname{.gbs},
                              \fname{.ay}, \fname{.sgc}, \fname{.vgm} \\
     Artist                 & \fname{.tta}, \fname{.spc}, \fname{.mmf}, \fname{.sid}, 
                              \fname{.rm}, \fname{.ra}, \fname{.rmvb}, \fname{.nsf}, 
                              \fname{.nsfe}, \fname{.sap}, \fname{.gbs}, \fname{.ay},
                              \fname{.sgc}, \fname{.vgm} \\
     Album                  & \fname{.spc}, \fname{.sid}, \fname{.nsf}, \fname{.nsfe},
                              \fname{.gbs}, \fname{.ay}, \fname{.sgc}, \fname{.vgm} \\
     Genre                  & \fname{.tta}, \fname{.spc}, \fname{.sap} \\
     Disc                   & \fname{.tta} \\
     Track                  & \fname{.tta} \\
     Year                   & \fname{.spc}, \fname{.sid}, \fname{.sap} \\
     Composer               & \fname{.mmf} \\
     Comment                & \fname{.spc}, \fname{.rm}, \fname{.ra}, \fname{.rmvb},
                              \fname{.vgm} \\
     Albumartist            & None \\
     Grouping               & None \\
    \end{rbtabular}
    
    \subsection{Limitations of metadata handling}
    \begin{enumerate}
        \item Multiple tags (e.g. for Genre) are not supported. The first tag 
              item of a set of multiple tags is used.
        \item Only one tag type is supported for each audio format.
    \nopt{clipv1,c200v2}{
        \item Overall there are 900 bytes available to load metadata strings.
        \item The maximum size of each metadata item (e.g. Artists) is limited 
              to 240 bytes.
    }
    \opt{clipv1,c200v2}{
        \item Overall there are 300 bytes available to load metadata strings.
        \item The maximum size of each metadata item (e.g. Artists) is limited 
              to 90 bytes.
    }
    \end{enumerate}
}


\opt{lcd_bitmap}{\chapter{\label{ref:album_art}Album Art}
Rockbox allows you to put the album art, or another image related to the music
on your \dap{} to display it in the PictureFlow plugin\opt{albumart}{ or in the
theme}. For this feature to work, there are a few requirements.

\section{Limitations}

\opt{albumart}{%
   Rockbox supports embedded album art only for some specific formats, see
  \reference{ref:featureset_for_generic_metadata_tags} for full details. It additionally
  supports loading images located on the \disk{}. PictureFlow is currently unable to
  use embedded album art.
}%
\nopt{albumart}{%
   Rockbox currently only supports loading images located on the
   \disk{} for use in PictureFlow.
}%
The image files must be in either BMP or JPEG format\opt{albumart}{, while embedded
album art is currently limited to JPEG.  Embedded JPEG images must not be
unsynchronized}. Rockbox does not support RLE-compressed BMP files, nor does it
support progressive and multi-scan JPEG files.
JPEG files must consist of a single scan with interleaved components, 
as progessive and multi-scan images require much more memory to decode.

\section{Where to put album art}

The pictures can be named a number of different ways, and placed to a number of
different locations. You can have pictures specific to the file or the album
or use a generic picture. You can place the picture in the same directory
as the file, in the parent directory or in a fixed directory named
\fname{/.rockbox/albumart/}. The order Rockbox uses when looking for a picture
is as follows (a list in braces means that those file extensions are tried in
that order):

\begin{enumerate}
\item  embedded (JPEG images in ID3v2 or MP4 tags only)
\item  \fname{./filename.\{jpeg,jpg,bmp\}}
\item  \fname{./albumtitle.\{jpeg,jpg,bmp\}}
\item  \fname{./cover.\{jpeg,jpg,bmp\}}
\item  \fname{./folder.jpg}
\item  \fname{/.rockbox/albumart/albumartist-albumtitle.\{jpeg,jpg,bmp\}}
\item  \fname{../albumtitle.\{jpeg,jpg,bmp\}}
\item  \fname{../cover.\{jpeg,jpg,bmp\}}
\end{enumerate}

The following characters will be replaced with an underscore (\_) when looking
for albumtitle.bmp or albumartist-albumtitle.bmp: \textbackslash{} / : <
> ? * |. Doublequotes will be replaced by single quotes.
If no album artist is set, artist will be used instead. See \wikilink{AlbumArt}
in the wiki for programs that will help you automate the process of putting
album art on your \dap{}.
}

% $Id$ %
\chapter{\label{ref:wps_tags}Theme Tags}
Themeing is discussed in detail in section \reference{ref:ConfiguringtheWPS},
what follows is a list of the available tags.

\note{The ``bar-type tags'' (such as \%pb, \%pv, \%bl etc.) can be further
  themed -- see \reference{ref:bar_tags}.}

\section{Status Bar}
\begin{tagmap}
  \config{\%we} & Display Status Bar\\
  \config{\%wd} & Hide Status Bar\\
  \config{\%wi} & Display the inbuilt Status Bar in the current viewport\\
\end{tagmap}
These tags override the player setting for the display of the status bar.
They must be noted on their own line (which will not be shown in the WPS).

\section{Hardware Capabilities}
\begin{tagmap}
    \config{\%cc} & Check for presence of a real time clock, returns ``c''
                   when used unconditionally\\
    \config{\%tp} & Does this target have a radio?\\
    \config{\%Tp} & Indicates that the target has a touchscreen\\
\end{tagmap}
With the above tags it is possible to find out about the presence of certain
hardware and make the theme adapt to it. This can be very useful for designing
a theme that works on multiple targets with differing hardware capabilities, e.g.
targets that do and do not have a clock. When used conditionally, the ``true''
branch is completely ignored if it does not apply.

Example:
\config{\%?cc<\%cH:\%cM|No clock detected>}


\section{Information from the track tags}
  \begin{tagmap}
    \config{\%ia} & Artist\\
    \config{\%ic} & Composer\\
    \config{\%iA} & Album Artist\\
    \config{\%id} & Album Name\\
    \config{\%iG} & Grouping\\
    \config{\%ig} & Genre Name\\
    \config{\%in} & Track Number\\
    \config{\%it} & Track Title\\
    \config{\%iC} & Comment\\
    \config{\%iv} & ID3 version (1.0, 1.1, 2.2, 2.3, 2.4, or empty if not an ID3 tag)\\
    \config{\%iy} & Year\\
    \config{\%ik} & Disc Number\\
  \end{tagmap}
Remember that this information is not always available, so use the 
conditionals to show alternate information in preference to assuming.

These tags, when written with a capital ``I'' (e.g. \config{\%Ia} or \config{\%Ic}),
show the information for the next song to be played.

\nopt{lcd_charcell}{
  \section{Viewports}
    \begin{tagmap}
      \nopt{lcd_non-mono}{%
        \config{\%V(x,y,[width],\tabnlindent[height],[font])}
        & See section \ref{ref:Viewports}\\}

      \nopt{lcd_color}{\opt{lcd_non-mono}{%
        \config{\%V(x,y,[width],\tabnlindent[height],[font])}\newline
        \config{\%Vf([fgshade])}\newline
        \config{\%Vb([bgshade])}
        & See section \ref{ref:Viewports}\\}}

      \opt{lcd_color}{%
        \config{\%V(x,y,[width],\tabnlindent[height],[font])}\newline
        \config{\%Vf([fgcolour])}\newline
        \config{\%Vb([bgcolour])}\newline
        \config{\%Vg(start,end \tabnlindent[,text])}
        & See section \ref{ref:Viewports}\\}

      \opt{lcd_non-mono}{%
        \config{\%Vs(mode[,param])}
        & See section \ref{ref:Viewports}\\}

      \config{\%Vl('identifier',\newline\dots)} & Preloads a viewport for later
      display. `identifier' is a single lowercase letter (a-z) and the `\dots'
      parameters use the same logic as the \%V tag explained above.\\

      \config{\%Vd('identifier')} & Display the `identifier' viewport. E.g.
      \config{\%?C<\%Vd(a)|\%Vd(b)>}
      will show viewport `a' if album art is found, and `b' if it isn't.\\

      \config{\%Vi('label',\dots)} &
      Declare a Custom UI Viewport. The `\dots' parameters use the same logic as
      the \config{\%V} tag explained above. See section \ref{ref:Viewports}.\\ 

      \config{\%VI('label')} & Set the Info Viewport to use the viewport called
      label, as declared with the previous tag.\\

      \config{\%VB} & Draw this viewport on the backdrop layer.\\
    \end{tagmap}

  \section{Additional Fonts}
    \begin{tagmap}
      \config{\%Fl('id',filename)} & See section \ref{ref:multifont}.\\
    \end{tagmap}

  \section{Misc Coloring Tags}
    \begin{tagmap}
      \config{\%dr(x,y,width,height,[color1,color2])} & Color a rectangle. \\
    \end{tagmap}
      width and height can be - to fill the viewport. If no color is
      specified the viewports foreground color will be used. If two
      colors are specified it will do a gradient fill.
}

\section{Power Related Information}
  \begin{tagmap}
    \config{\%bl} & Numeric battery level in percents.
                    Can also be used in a conditional: 
                    \config{\%?bl<-1|0|1|2|\ldots|N>},
                    where the value $-1$ is used when the battery level isn't
                    known (it usually is). The value $N$ is only used when the
                    battery level is exactly 100 percent.
                    An image can also be used, the proportion of the image
                    shown corresponds to the battery level:
                    \config{\%bl(x,y,[width],[height],image.bmp)}\\
    \config{\%bv} & The battery level in volts\\
    \config{\%bt} & Estimated battery time left\\
    \config{\%bp} & ``p'' if the charger is connected (only on targets
                    that can charge batteries)\\
    \config{\%bc} & ``c'' if the unit is currently charging the battery (only on
                    targets that have software charge control or monitoring)\\
    \config{\%bs} & Remaining time of the sleep timer (if it is set)\\
  \end{tagmap}

\section{Information about the file}
  \begin{tagmap}
    \config{\%fb} & File Bitrate (in kbps)\\
    \config{\%fc} & File Codec (e.g. ``MP3'' or ``FLAC'').
           This tag can also be used in a conditional tag:
           \config{\%?fc<mp1|mp2|mp3|aiff|wav|ogg|\newline
           flac|mpcsv7|a52|wavpack|alac|aac|shn|sid|adx|nsf|\newline
           speex|spc|ape|wma|wmpapro|mod|sap|realaudiocook|\newline
           realaudioaac|realaudioac3|realaudioatrac3|cmc|\newline
           cm3|cmr|cms|dmc|dlt|mpt|mpd|rmt|tmc|tm8|tm2|\newline
           omaatrac3|smaf|au|vox|wave64|tta|wmavoice|mpcsv8|\newline
           aache|ay|gbs|hes|sgc|vgm|kss|unknown>}.
                  The codec order is as shown above.\\
    \config{\%ff} & File Frequency (in Hz)\\
    \config{\%fk} & File Frequency (in kHz)\\
    \config{\%fm} & File Name\\
    \config{\%fn} & File Name (without extension)\\
    \config{\%fp} & File Path\\
    \config{\%fs} & File Size (in Kilobytes)\\
    \config{\%fv} & ``(avg)'' if variable bit rate or empty string if constant bit rate\\
    \config{\%d(N)} & N-th segment from the end of the file's directory
                       (N can be 1, 2, 3, \dots)\\
  \end{tagmap}
Example for the \config{\%d(N)} commands: If the file is
``/Rock/Kent/Isola/11 - 747.mp3'', \config{\%d(1)} is ``Isola'', 
\config{\%d(2)} is ``Kent'' and \config{\%d(3)} is ``Rock''.

These tags, when written with the first letter capitalized (e.g. \config{\%Fn}
or \config{\%D(2)}), produce the information for the next file to be played.

\section{Playlist/Song Info}
  \begin{tagmap}
    \config{\%pb} & Progress Bar.
    \opt{player}{
            This will display a one character ``cup''
            that empties as the time progresses.}
    \opt{lcd_bitmap}{
           This will replace the entire line with a progress bar.
           You can set the position, width and height of the progressbar
           (in pixels) and load a custom image for it:
           \config{\%pb(x,y,[width],[height],image.bmp)}} \\
    \opt{player}{%
    \config{\%pf} & Full-line progress bar \& time display\\
    }%
    \config{\%px} & Percentage played in song\\
    \config{\%pc} & Current time in song\\
    \config{\%pe} & Total number of playlist entries\\
    \nopt{player}{%
    \config{\%pm} & Peak Meter. The entire line is used as %
                    volume peak meter.\\%
    \config{\%pL} & Peak meter for the left channel. Can be used as a value, %
            a conditional tag or a bar tag.\\
    \config{\%pR} & Peak meter for the right channel. Can be used as a value, %
            a conditional tag or a bar tag.\\
    }%
    \config{\%pn} & Playlist name (without path or extension)\\
    \config{\%pp} & Playlist position\\
    \config{\%pr} & Remaining time in song\\
    \config{\%ps} & ``s'' if shuffle mode is enabled\\
    \config{\%pt} & Total track time\\
    \config{\%pv} & Current volume (in dB). Can also be used in a conditional:
           \config{\%?pv<Mute|\ldots|0 dB|Above 0 dB>}\newline
           Mute is 0\% volume, \ldots is the values between Mute and max, 0 dB is max volume, and Above 0 dB is amplified volume
         \opt{lcd_bitmap}{This can also be used like \%pb to provide a continuous scale:
         \config{\%pv(x,y,[width],[height],image.bmp)}} \\
    \config{\%pS} & Track is starting. An optional number gives how many seconds
         the tag remains true for after the start of the track. The default is
         10 seconds if no number is specified.
         \config{\%?pS(7)<in the first 7 seconds of track|in
           the rest of the track>}\\
    \config{\%pE} & Track is ending. An optional number gives how many seconds
         before the end of the track the tag becomes true. The default is
         10 seconds if no number is specified.
         \config{\%?pE(7)<in the last 7 seconds of track|in
           the rest of the track>}\\
    \config{\%Sp} & Current playback pitch\\
  \end{tagmap}
  
\section{Playlist Viewer}
  \begin{tagmap}
    \config{\%Vp(start,code to render)} & Display the playlist viewer in
            the current viewport.\\
  \end{tagmap}

  \begin{itemize}
    \item `start' is the offset relative to the currently playing track for the
    playlist to display from (0 the current track, 1 is the next track, etc.).
    \item `code to render' is a line of skin code which will be displayed for
    each line in the viewer. All text tags are supported (including conditionals
    and sublines)
  \end{itemize}

  The entire viewport will be used, so don't expect other tags in the same
  viewport to work well.  Supported tags are \%pp, all tags starting with \%i,
  most tags starting with \%f, \%pt and \%s.\\

  Example: \config{\%Vp(1,\%pp - \%it,\%pp - \%fn)} -- Display the playlist
  position, then either the track title (from the tags) or
  the filename. The viewer will display as many tracks as will fit in the
  viewport.

\section{Runtime Database}
  \begin{tagmap}
    \config{\%rp} & Song playcount\\
    \config{\%rr} & Song rating (0-10). This tag can also be used in a conditional tag: %
           \config{\%?rr<0|1|2|3|4|5|6|7|8|9|10>}\\
    \config{\%ra} & Autoscore for the song\\
  \end{tagmap}

\opt{swcodec}{
\section{Sound (DSP) settings}
  \begin{tagmap}
    \config{\%Sp} & Current playback pitch \\
  \opt{swcodec}{
    \config{\%xf} & Crossfade setting, in the order: Off, Auto Skip, Man Skip,
           Shuffle, Shuffle and Man Skip, Always\\
    \config{\%rg} & ReplayGain value in use (x.y~dB). If used as a conditional,
           Replaygain type in use: \config{\%?rg<Off|Track%
           |Album|TrackShuffle|AlbumShuffle%
           |No tag>}\\
  }
  \end{tagmap}
}

\section{Hold}
    \begin{tagmap}
        \config{\%mh} & ``h'' if the main unit keys are locked\\
        \opt{remote_button_hold}{%
            \config{\%mr} & ``r'' if the remote keys are locked\\
        }
    \end{tagmap}

\section{Virtual LED}
  \begin{tagmap}
    \config{\%lh} & ``h'' if the \disk\ is accessed\\
  \end{tagmap}

\section{Repeat Mode}
  \begin{tagmap}
    \config{\%mm} & Repeat mode, 0-4, in the order: Off, All, One, Shuffle, A-B\\
  \end{tagmap}
Example: \config{\%?mm<Off|All|One|Shuffle|A-B>}

\section{Playback Mode}
  \begin{tagmap}
    \config{\%mp} & Play status, 0-4, in the order: Stop, Play, Pause, 
           Fast Forward, Rewind, Recording, Recording paused, FM Radio playing,
           FM Radio muted\\
  \end{tagmap}
Example: \config{\%?mp<Stop|Play|Pause|Ffwd|Rew|Rec|Rec pause|FM|FM pause>}

\section{Current Screen}
  \begin{tagmap}
    \config{\%cs} & The current screen, 1-20, in the order shown below\\
  \end{tagmap}

\begin{table}
  \begin{rbtabular}{.75\textwidth}{lX}%
  {\textbf{Number} & \textbf{Screen}}{}{}
    1 & Menus \\
    2 & WPS \\
    3 & Recording screen \\
    4 & FM Radio screen \\
    5 & Current Playlist screen \\
    6 & Settings menus \\
    7 & File browser \\
    8 & Database browser \\
    9 & Plugin browser \\
    10 & Quickscreen \\
    11 & Pitchscreen \\
    12 & Setting chooser \\
    13 & Playlist Catalogue Viewer \\
    14 & Plugin \\
    15 & Context menu \\
    16 & System Info screen \\ 
    17 & Time and Date Screen \\
    18 & Bookmark browser \\
    19 & Shortcuts menu \\
    20 & Track Info screen \\
  \end{rbtabular}
\end{table}

The tag can also be used as the switch in a conditional tag. For players without
certain capabilities (e.g. no FM radio) some values will never be returned.

Examples:

\config{You are in the \%?cs<Main menu|WPS|Recording screen|FM Radio screen>}

\config{\%?if(\%cs, =, 2)<This is the WPS>}

\section{List Title (\fname{.sbs} only)}
  \begin{tagmap}
    \config{\%Lt} & Title text. Should be used in a conditional so that non-list
      screens don't show a title when they shouldn't\\
    \config{\%Li} & Title icon. This uses the same order as custom icons (see
      \wikilink{CustomIcons} in the wiki) except that here \config{0} is ``no
      icon''\\
  \end{tagmap}

  This tag can be used to give custom formatting to list titles.
  Define a viewport with the font and formatting desired, and then use
  \config{\%?Lt<\%Lt>} to display the title within the
  viewport.  If \config{\%Lt} is present anywhere in the \fname{.sbs}, then the
  \config{\%Vi} viewport will not show the title.

\section{Changing Volume}
  \begin{tagmap}
    \config{\%mv(t)} & ``v'' if the volume is being changed\\
  \end{tagmap}

The tag produces the letter ``v'' while the volume is being changed and some
amount of time after that, i.e. after the volume button has been released. The
optional parameter \config{t} specifies that amount of time, in seconds. If it
is not specified, 1 second is assumed.

The tag can be used as the switch in a conditional tag to display different things
depending on whether the volume is being changed. It can produce neat effects
when used with conditional viewports.

Example: \config{\%?mv(2.5)<Volume changing|\%pv>}

The example above will display the text ``Volume changing'' if the volume is
being changed and 2.5 seconds after the volume button has been released. After
that, it will display the volume value.

\section{Settings}
  \begin{tagmap}
    \config{\%St(<setting\tabnlindent{}name>)} & The value of the Rockbox
             setting with the specified name. See \reference{ref:config_file_options}
             for the list of the available settings.\\
    \config{\%St(...)} & Draw a bar using from the setting.
            See \reference{ref:bar_tags} for details.\\
  \end{tagmap}

Examples:
\begin{enumerate}
\item As a simple tag: \config{\%St(skip length)}
\item As a conditional: \config{\%?St(eq enabled)<Eq is enabled|Eq is disabled>}
\end{enumerate}


\opt{lcd_bitmap}{
\section{\label{ref:wps_images}Images}
  \begin{tagmap}
    \nopt{archos}{%
    \config{\%X(filename.bmp)}
        & Load and set a backdrop image for the WPS.
          This image must be exactly the same size as your LCD.\\
    }%
    \config{\%x(n,filename[,x,y])}
        & Load and display an image\newline
          \config{n}: image ID for later referencing in \config{\%xd}\newline
          \config{filename}: file name relative to \fname{/.rockbox/} and including ``.bmp''\newline
          \config{x}: x coordinate (defaults to 0 if both x and y are not specified)\newline
          \config{y}: y coordinate. (defaults to 0 if both x and y are not specified)\\
    \config{\%xl(n,filename,[x,y],\tabnlindent[nimages])}
        & Preload an image for later display (useful for when your images are displayed conditionally).\newline
          \config{n}: image ID for later referencing in \config{\%xd}\newline
          \config{filename}: file name relative to \fname{/.rockbox/} and including ``.bmp''\newline
            If the filename is ``\_\_list\_icons\_\_'' the list icon bitmap will be used instead\newline
          \config{x}: x coordinate (defaults to 0 if both x and y are not specified)\newline
          \config{y}: y coordinate. (defaults to 0 if both x and y are not specified)\newline
          \config{nimages}: (optional) number of sub-images (tiled vertically, of the same height)
          contained in the bitmap. Default is 1.\\
    \config{\%xd(n[i] [,tag] [,offset])} & Display a preloaded image.
          \config{n}: image ID  as it was specified in \config{\%x} or \config{\%xl}\newline
          \config{i}: (optional) number of the sub-image to display (a-z for 1-26 and A-Z for 27-52). 
          (ignored when \config{tag} is used). Only useable if the ID is a single letter.
          By default the first (i.e. top most) sub-image will be used.\newline
          \config{tag}: (optional) Another tag to calculate the subimage from e.g \config{\%xd(A, \%mh)} would
          use the first subimage when \config{\%mh} is on and the second when it is off\newline
          \config{offset}: (optional) Add this number to the value from the \config{tag} when 
          chosing the subimage (may be negative)\\
    \config{\%x9(n)}
        & Display an image as a 9-patch bitmap covering the entire viewport.\newline
          9-patch images are bitmaps split into 9 segments where the four corners
          are unscaled, the four middle sections are scaled along one axis and the middle
          section is scaled on both axis.\newline
          \config{n}: image ID\\
          
  \end{tagmap}

Examples:
\begin{enumerate}
\item Load and display the image \fname{/.rockbox/bg.bmp} with ID ``a'' at 37, 109:\\
\config{\%x(a,bg.bmp,37,109)}
\item Load a bitmap strip containing 5 volume icon images (all the same size)
with image ID ``M'', and then reference the individual sub-images in a conditional:\\
\config{\%xl(M,volume.bmp,134,153,5)}\\
\config{\%?pv<\%xd(Ma)|\%xd(Mb)|\%xd(Mc)|%
\%xd(Md)|\%xd(Me)>}
\end{enumerate}


\note{
  \begin{itemize}
  \item The images must be in BMP format
  \item The image tag must be on its own line
  \item The ID is case sensitive
  \item The size of the LCD screen for each \dap{} varies. See table below 
        for appropriate sizes of each device. The x and y coordinates must 
        respect each of the \daps{} limits.
  \end{itemize}
}
}

\opt{albumart}{
\subsection{How to display the album art}

Once the album art files are present on your \dap, they can be displayed as
follows.

  \begin{tagmap}
    \config{\%Cl(x,y,[maxwidth],\tabnlindent[maxheight],\tabnlindent{}hor\_align,\tabnlindent{}vert\_align)}
        & Define the settings for album art\newline
          \config{x}: x coordinate\newline
          \config{y}: y coordinate\newline
          \config{maxwidth}: Maximum height\newline
          \config{maxheight}: Maximum width\newline
          \config{hor\_align}: Horizontal alignment, enter as `l', `c' or `r' for
          left, centre or right. Centre is default\newline
          \config{vert\_align}: Vertical alignment, enter as `t', `c' or `b' for
          top, centre or bottom. Centre is default\\
    \config{\%Cd}  & Display the album art as configured. \\
    \config{\%C}  & Use in a conditional to determine if an image is available. \\
  \end{tagmap}

The picture will be rescaled, preserving aspect ratio to fit the given
\config{maxwidth} and \config{maxheight}. If the aspect ratio doesn't match the
configured values, the picture will be placed according to the alignment flags.

Examples:
\begin{enumerate}
  \item Load albumart at position 20,40 and display it without resizing:\\
  \config{\%Cl(20,40,,)}
  \item Load albumart at position 0,20 and resize it to be at most 100$\times$100
        pixels. If the image isn't square, align it to the bottom-right
        corner:\\
  \config{\%Cl(0,20,100,100,r,b)}
\end{enumerate}

For general information where to put album art see \reference{ref:album_art}.
}

\opt{radio}{
\section{FM Radio}
  \begin{tagmap}
    \config{\%tt} & Is the tuner tuned?\\
    \config{\%tm} & Scan or preset mode? Scan is ``true'', preset is ``false''.\\
    \config{\%ts} & Is the station in stereo?\\
    \config{\%ta} & Minimum frequency (region specific) in MHz.\\
    \config{\%tb} & Maximum frequency (region specific) in MHz.\\
    \config{\%tf} & Current frequency in MHz.\\
    \config{\%Ti} & Current preset id, i.e. 1-based number of the preset
      within the presets list (usable in playlist viewer).\\
    \config{\%Tn} & Current preset name (usable in playlist viewer).\\
    \config{\%Tf} & Current preset frequency (usable in playlist viewer).\\
    \config{\%Tc} & Preset count, i.e. the number of stations in the current
      preset list.\\
    \config{\%tx} & Is RDS available?\\
    \config{\%ty} & RDS name.\\
    \config{\%tz} & RDS text.\\
    \config{\%tr} & Signal strength (RSSI). Can be used in a conditional or as
      a bar.\\
  \end{tagmap}

It is also possible to show ``Radio Art'' which can be used to display images
associated with presets. The tags are exactly the same as for album art,
described above. Images need to be placed in \fname{/.rockbox/fmpresets/},
and must have the same name as the preset. They need to be in either
\fname{.bmp} or \fname{.jpg} format, and the radio must be in preset mode
and tuned to a preset (and not recording) in order for them to be shown.
}

\section{Alignment and language direction}
  \begin{tagmap}
    \config{\%al} & Align the text left\\
    \config{\%aL} & Align the text left, or to the right if RTL language is in use\\
    \config{\%ac} & Centre the text\\
    \config{\%ar} & Align the text right\\
    \config{\%aR} & Align the text right, or to the left if RTL language is in use\\
    \config{\%ax} & The next tag should follow the set language direction. When
                    prepended to a viewport declaration, the viewport will
                    be horizontally mirrored if the user language is set to
                    a RTL language. Currently the \%Cl, \%V and \%Vl tags
                    support this.\\
    \config{\%Sr} & Use as a conditional to define options for left to right, or
                    right to left languages. \%?Sr<RTL|LTR>\\
  \end{tagmap}
  
All alignment tags may be present in one line, but they need to be in the 
order left -- centre -- right. If the aligned texts overlap, they are merged.

Example: \config{\%ax\%V(\dots)}

\section{Conditional Tags}

\begin{tagmap}
\config{\%?xx<true|false>}
    & If / Else: Evaluate for true or false case \\
\config{\%?xx<alt1|alt2|\tabnlindent{}alt3|\dots|else>}
    & Enumerations: Evaluate for first / second / third / \dots / last condition \\
\config{\%if(tag, operator, operand, [option count])}
    & Allows very simple comparisons with other tags.\newline
      \config{tag}: the tag to check against.\newline
      \config{operator}: the comparison to perform - possible options are =, !=,
        >, >=, <, <=\newline
      \config{operand}: either a second tag, a number, or text.\newline
      \config{[option count]}: optional parameter used to select which parameter
        of a tag to use when the tag has multiple options, e.g. \%?pv<a|b|c|d>\\
\config{\%and(tag1, tag2, ..., tagN)}\newline
    & Logical ``and'' operator. Will be evaluate to true if all the tag parameters are true.\\
\config{\%or(tag1, tag2, ..., tagN)}\newline
    & Logical ``or'' operator. Will be evaluate to true if any of the tag parameters are true.\\
\end{tagmap}

Examples of the \%if tag:\\

\config{\%?if(\%pv, >=, 0)<Clipping possible|Volume OK>} will display ``Clipping
  possible'' if the volume is higher than or equal to 0 dB, ``Volume OK'' if it
  is lower.\\

\config{\%?if(\%ia, =, \%Ia)<same artist>} -- this artist and the next artist
  are the same.\\

\note{When performing a comparison against a string tag such as \%ia, only = and
  != work, and the comparison is not case sensitive.}

\section{Subline Tags}

\begin{tagmap}
\config{\%t(time)}
    & Set the subline display cycle time (\%t(5) or \%t(3.4) formats) \\
\config{;}
    & Split items on a line into separate sublines \\
\end{tagmap}

Allows grouping of several items (sublines) onto one line, with the
display cycling round the defined sublines. See
\reference{ref:AlternatingSublines} for details. 

\opt{rtc}{
\section{Time and Date}
    \begin{tagmap}
      \config{\%cd}          & Day of month from 01 to 31\\
      \config{\%ce}          & Zero padded day of month from 1 to 31\\
      \config{\%cf}          & A conditional for 12/24 hour format.\newline
                               \config{\%?cf<24 hour stuff|12 hour stuff>}\\
      \config{\%cH}          & Zero padded hour from 00 to 23 (24 hour format)\\
      \config{\%ck}          & Hour from 0 to 23 (24 hour format)\\
      \config{\%cI}          & Zero padded hour from 01 to 12 (am/pm format)\\
      \config{\%cl}          & Hour from 1 to 12 (am/pm format)\\
      \config{\%cm}          & Month from 01 to 12\\
      \config{\%cM}          & Minutes\\
      \config{\%cS}          & Seconds\\
      \config{\%cy}          & 2-digit year\\
      \config{\%cY}          & 4-digit year\\
      \config{\%cP}          & Capital AM/PM\\
      \config{\%cp}          & Lowercase am/pm\\
      \config{\%ca}          & Weekday name\\
      \config{\%cb}          & Month name\\
      \config{\%cu}          & Day of week from 1 to 7, 1 is Monday\\
      \config{\%cw}          & Day of week from 0 to 6, 0 is Sunday\\
    \end{tagmap}
}

\section{Text Translation}
  \begin{tagmap}
    \config{\%Sx(English)}
    & Display the translation of ``English'' in the current language\\
  \end{tagmap}
  
  \begin{itemize}
    \item ``English'' must be a phrase used in the language file.
    \item It should match the \config{Source:} line in the language file.
  \end{itemize}

  \note{checkwps cannot verify that the string is correct, so please check on
    either the simulator or on target.}


\opt{touchscreen}{
  \section{Touchscreen Areas}
    \begin{tagmap}
      \config{\%T(x,y,width,\tabnlindent{}height, action, [options])}
      & Invoke the action specified when the user presses in the defined
      touchscreen area.\\
    \end{tagmap}

  Possible actions are:

  \begin{description}
    \item[none] -- Do nothing.
    \item[play] -- Play/pause playback.
    \item[stop] -- Stop playback and exit the WPS.
    \item[prev] -- Previous track/item.
    \item[next] -- Next track/item.
    \item[wps\_prev] -- Previous track.
    \item[wps\_next] -- Next track.
    \item[ffwd] -- Seek forwards in the track.
    \item[rwd] -- Seek backwards in the track.
    \item[progressbar] -- Seek to the appropriate position in the track based on the touch.
    \item[shuffle] -- Toggle shuffle mode.
    \item[repmode] -- Cycle through the repeat modes.
    \item[volume] -- Set the volume to the appropriate level based on the touch.
    \item[voldown] -- Decrease the volume by one step.
    \item[volup] -- Increase the volume by one step.
    \item[mute] -- Un/Mute playback.
    \item[createbookmark] -- Create a bookmark in the currently-playing track.
    \item[hotkey] -- Performs the action assigned to the hotkey (see Hotkeys section).
    \item[menu] -- Go to the main menu.
    \item[browse] -- Go back to the file browser or database.
    \item[resumeplayback] -- Go back to the last music screen (WPS or radio screen).
    \item[quickscreen] -- Go to the quickscreen.
    \item[contextmenu] -- Open the context menu.
    \item[playlist] -- Go to the playlist viewer.
    \item[listbookmarks] -- List the bookmarks for the currently-playing directory or playlist.
    \item[trackinfo] -- Open the track info viewer.
    \item[pitch] -- Open the pitchscreen.
    \item[setting\_inc] -- Increment the subsequently specified setting (e.g
      \config{\%T(0,0,32,32, setting\_inc, volume)} increases the volume by one step).
    \item[setting\_dec] -- Decrement the subsequently specified setting (e.g
      \config{\%T(0,0,32,32, setting\_dec, volume)} decreases the volume by one step).
    \item[setting\_set] -- Set the subsequently specified setting to a specific value (e.g
      \config{\%T(0,0,32,32, setting\_set, volume, 0)} sets the volume to 0).
    \item[lock] -- Soft locks the touchscreen.  All touch areas are disabled except for
        areas with the lock action or ones that have the allow\_while\_locked option (see below).
  \end{description}
  Any (or muliple) of the following options can be used after the action is specified
  \subsection{Options}
  \begin{description}
    \item[repeat\_press] -- This area will trigger mulitple times when held (i.e for seeking)
    \item[long\_press] -- This area will trigger once after it is held for a long press
    \item[reverse\_bar] -- Reverse the bars touch direction (i.e seek right to left)
    \item[allow\_while\_locked] -- Allows the area to be pressable when the
        skin is locked by the lock touch action
  \end{description}

  \section{Last Touchscreen Press}
    \begin{tagmap}
      \config{\%Tl} & Indicates that the touchscreen is pressed.\\
    \end{tagmap}
  This tag can be used to display text or images or a viewport when the
  touchscreen is pressed (like an On Screen Display). If you put a number
  straight after \%Tl it will be used as a timeout in seconds
  (e.g \%Tl(2.5) will give a 2.5 second timeout) between the touchscreen press
  being released and the tag going false. If no number is specified it will
  use a 1 second timeout.  It can also be used as a conditional, and can be
  used with conditional viewports.
}

\section{\label{ref:bar_tags}Bar Tags}
  Some tags can be used to display a bar which draws according to the value of
  the tag. To use these tags like a bar you need to use the following parameters
  (\%XX should be replaced with the actual tag).
  
\opt{touchscreen}{
    Volume and progress bars automatically create touch regions the same size
    as the bar (slightly larger actually). This can be disabled with the
    \config{notouch} option.
}

\begin{tagmap}
  \config{\%XX(x, y, width, height, [options])}
    & Draw the specified tag as a bar\newline
      \config{x}: x co-ordinate at which to start drawing the bar.\newline
      \config{y}: y co-ordinate at which to start drawing the bar (- to make the
        bar appear on the line of the tag, as if it was a text tag) .\newline
      \config{width}: width of the bar (- for the full viewport width).\newline
      \config{height}: height of the bar (- to set to the font height for
        horizontal bars and to the viewport height for vertical bars).\newline
      \config{options}: any of the options set out below.\\
\end{tagmap}

\subsection{Options}
\begin{description}
  \item[image] -- the next option is either the filename or image label to
    use for the fill image.
  \item[horizontal] -- force the bar to be drawn horizontally.
  \item[vertical] -- force the bar to be drawn vertically.
  \item[invert] -- invert the draw direction (i.e. right to left, or top to
    bottom).
  \item[slider] -- draw a preloaded image over the top of the bar so that
    the centre of the image matches the current position. This must be
    followed by the label of the desired image.
  \item[backdrop] -- draw a preloaded image under the bar. The full
    image will be displayed and must be the same size as the bar. 
    This must be followed by the label of the desired image.
  \item[nofill] -- don't draw the bar, only its frame (for use with the
     ``slider'' option).
  \item[noborder] -- don't draw the border for image-less bars, instead maximise
    the filling over the specified area. This doesn't work for bars which
    specify an image. 
  \item[nobar] -- don't draw the bar or its frame (for use with the
    ``slider'' option).
    \opt{touchscreen}{
        \item[notouch] -- don't create the touchregion for progress/volume bars.
    }
  \item[setting] -- Specify the setting name to draw the bar from (bar must be
    \%St type), the next param is the settings config name.
\end{description}

Example: \config{\%pb(0,0,-,-,-,nofill, slider, slider\_image, invert)} -- draw
a horizontal progressbar which doesn't fill and draws the image
``slider\_image'' which moves right to left.

\note{If the slider option is used, the bar will be shrunk so that the slider
  fits inside the specified width and height. Example: A 100px bar image with a
  16px slider image needs the bar to be 116px wide, and should be offset 8px
  left of the backdrop image to align correctly.}

\section{Other Tags}

\begin{tagmap}
  \config{\%ss(start, length, tag [,number]} & Get a substring from another tag.\\
\end{tagmap}
    Use this tag to get a substring from another tag.
\begin{description}
    \item[start] -- first character to take (0 being the start of the string, negative means from the end of the string)
    \item[length] -- length of the substring to return (- for the rest of the string)
    \item[tag] -- tag to get
    \item[number] -- OPTIONAL. if this is present it will assume the
            substring is a number so it can be used with conditionals. (i.e \config{\%cM}).
            0 is the first conditional option
\end{description}
    
\begin{tagmap}
  \config{\%(}           & The character `('\\
  \config{\%)}           & The character `)'\\
  \config{\%,}           & The character `,'\\
  \config{\%\%}          & The character `\%'\\
  \config{\%<}           & The character `<'\\
  \config{\%|}           & The character `|'\\
  \config{\%>}           & The character `>'\\
  \config{\%;}           & The character `;'\\
  \config{\%\#}          & The character `\#'\\
  \config{\%s}           & Indicate that the line should scroll. Can occur 
                           anywhere in a line (given that the text is 
                           displayed; see conditionals above). You can specify 
                           up to ten scrolling lines. Scrolling lines can not 
                           contain dynamic content such as timers, peak meters 
                           or progress bars.\\
\end{tagmap}


% $Id$ %
\chapter{\label{ref:config_file_options}Config file options}
\begin{center}
% define a local version of endhead, as using the output distinction adds
% an unwanted newline. endhead breaks with htlatex so we need to remove it
% for the html output.
\ifpdfoutput{\newcommand{\localendhead}{\endhead}}%
    {\newcommand{\localendhead}{}}
  \rowcolors{2}{tbloddrowbgcolor}{tblevenrowbgcolor}
  \begin{longtable}{>{\raggedright}p{.3\textwidth}>{\raggedright}p{.4\textwidth}p{.2\textwidth}}
    \toprule
    \rowcolor{tblhdrbgcolor}\tblhdrstrut\textbf{Setting} & \textbf{Allowed Values} & \textbf{Unit}\\
    \midrule\localendhead % endhead breaks with htlatex
    volume      & \opt{masd}{-78 to +18}%
                  \opt{masf}{-100 -to +12}%
                  \opt{iriverh100,iriverh300}{-84 to 0}%
                  \opt{ipodnano}{-72 to +6}%
                  \opt{ipodvideo,cowond2}{-89 to +6}%
                  \opt{iaudiom3,iaudiom5,iaudiox5}{-73 to +6}%
                  \opt{fuzeplus}{-99 to +6}%
                  \opt{e200,e200v2,ipodcolor,mrobe100,vibe500,ipodnano2g}{-74 to +6}%
                  \opt{samsungyh}{-128 to 0}%
                                        & dB\\
    bass        & \opt{masd}{-15 to +15}%
                  \opt{masf}{-12 to +12}%
                  \opt{iriverh100,iriverh300}{0 to +24}%
                  \opt{ipod,mrobe100}{-6 to +9}%
                  \opt{iaudiom3,iaudiom5,iaudiox5,e200,e200v2,vibe500,fuzeplus,%
                       samsungyh}{-24 to +24}%
                                        & dB\\
    treble      & \opt{masd}{-15 to +15}%
                  \opt{masf}{-12 to +12}%
                  \opt{iriverh100,iriverh300}{0 to +6}%
                  \opt{ipod,mrobe100}{-6 to +9}%
                  \opt{iaudiom3,iaudiom5,iaudiox5,e200,e200v2,vibe500,fuzeplus,%
                       samsungyh}{-24 to +24}%
                                        & dB\\
    balance         & -100 to +100      & \%\\
    channels        & stereo, mono, custom, mono left, mono right, karaoke
                                        & N/A\\
    stereo\_width   & 0 to 250          & \%\\
    shuffle         & on, off               & N/A\\
    repeat          & off, all, one, shuffle, ab
                                        & N/A\\
    play selected   & on, off           & N/A\\
    party mode      & on, off           & N/A\\
    scan min step   & 1, 2, 3, 4, 5, 6, 8, 10, 15, 20, 25, 30, 45, 60
                                        & s\\
    seek acceleration & very fast, fast, normal, slow, very slow & N/A\\
    antiskip        & 5s, 15s, 30s, 1min, 2min, 3min, 5min, 10min & N/A\\
    volume fade     & on, off           & N/A\\
    sort case       & on, off           & N/A\\
    show files      & all, supported, music, playlists & N/A\\
    show filename exts & off, on, unknown, view\_all & N/A\\
    follow playlist & on, off           & N/A\\
    playlist viewer icons
                    & on, off           & N/A\\
    playlist viewer indices
                    & on, off           & N/A\\
    playlist viewer track display
                    & track name,full path
                                        & N/A\\
    recursive directory insert
                    & on, off, ask      & N/A\\
    scroll speed    & 1 to 25           & Hz\\
    scroll delay    & 0 to 2500         & ms\\
    scroll step     & \fixme{devise a way to get ranges from config-*.h} & pixels\\
    screen scroll step & \fixme{devise a way to get ranges from config-*.h} & pixels\\
    Screen Scrolls Out Of View & on, off & N/A\\
    bidir limit     & 0 to 200          & \% screen\\
    scroll paginated & on, off & N/A\\
    hold\_lr\_for\_scroll\_in\_list & on, off & N/A\\
    \opt{lcd_bitmap}{
      show path in browser & off, current directory, full path & N/A\\
    }
    contrast        & 0 to 63           & N/A\\
    backlight timeout
                    & off, on, 1, 2, 3, 4, 5, 6, 7, 8, 9, 10, 15, 20, 25, 30,
                      45, 60, 90, 120        & s\\
    backlight timeout plugged
                    & off, on, 1, 2, 3, 4, 5, 6, 7, 8, 9, 10, 15, 20, 25, 30,
                      45, 60, 90, 120        & s\\
    backlight filters first keypress & on, off & N/A\\
    backlight on button hold & normal, off, on & N/A\\
    caption backlight & on, off & N/A\\
    brightness      & \fixme{devise a way to get ranges from config-*.h} & N/A\\
    disk spindown   & 3 to 254          & s\\
    battery capacity & \fixme{devise a way to get ranges from config-*.h} & mAh\\
    \opt{battery_types}{
      battery type  & alkaline, nimh    & N/A\\
    }
    \opt{HAVE_CAR_ADAPTER_MODE}{
      car adapter mode & on, off & N/A\\
    }
    \opt{accessory_supply}{
      accessory power supply & on, off & N/A\\
    }
    \opt{usb_hid}{
        usb hid & on, off & N/A\\
        usb keypad mode
                    & multimedia, presentation, browser\opt{usb_hid_mouse}{, mouse}& N/A\\
    }
    \opt{multidrive_usb}{
        usb skip first drive & on, off & N/A\\
    }
    \opt{quickscreen}{
        qs top & any setting name, - for none & N/A\\
        qs bottom & any setting name, - for none & N/A\\
        qs left & any setting name, - for none & N/A\\
        qs right & any setting name, - for none & N/A\\
        shortcuts instead of quickscreen & off, on & N/A\\
    }

    idle poweroff   & off, 1, 2, 3, 4, 5, 6, 7, 8, 9, 10, 15, 30, 45, 60
                                        & min\\
    sleeptimer duration   & 5 to 300 (in steps of 5)
                                        & min\\
    sleeptimer on startup & off, on     & N/A\\
    keypress restarts sleeptimer & off, on & N/A\\
    max files in playlist & 1000 - 32000 & N/A\\
    max files in dir & 50 - 10000       & N/A\\
    lang            & /path/filename.lng & N/A\\
    wps             & /path/filename.wps & N/A\\
    autocreate bookmarks
                    & off, on           & N/A\\
    autoload bookmarks
                    & off, on           & N/A\\
    use most-recent-bookmarks
                    & off, on           & N/A\\
    pause on headphone unplug & off, pause, pause and resume & N/A\\
    rewind duration on pause & 0 to 15  & s\\
    disable autoresume if phones not present & off, on & N/A\\
    Last.fm Logging & off, on           & N/A\\
    talk dir        & off, number, spell& N/A\\
    talk dir clip   & off, on           & N/A\\
    talk file       & off, number, spell& N/A\\
    talk file clip  & off, on           & N/A\\
    talk filetype   & off, on           & N/A\\
    talk menu       & off, on           & N/A\\
    Announce Battery Level & off, on    & N/A\\
    \opt{hotkey}{
    hotkey wps      & off, view playlist, show track info,
        pitchscreen, open with, delete  & N/A\\
    \nopt{touchscreen}{hotkey tree     & off, open with, delete, insert,
        insert shuffled                 & N/A\\}
    }
    sort files      & alpha, oldest, newest, type & N/A\\
    sort dirs       & alpha, oldest, newest & N/A\\
    sort interpret number & digits, numbers & N/A\\
    tagcache\_autoupdate
                    & on, off           & N/A\\
    warn when erasing dynamic playlist
                    & on, off           & N/A\\
    cuesheet support
                    & on, off           & N/A\\
    folder navigation & off, on, random & N/A\\
    constrain next folder & off, on     & N/A\\
    gather runtime data & off, on       & N/A\\
    \opt{usb_charging_enable}{
      usb charging  & on, off, force    & N/A\\
    }
    skip length     & outro, track, 1s, 2s, 3s, 5s, 7s, 10s, 15s, 20s, 1min,
            90s, 2min, 3min, 5min, 10min, 15min & N/A\\
    prevent track skip
                    & on, off           & N/A\\
    start in screen & previous, root, files, dB, wps, menu,
      \opt{recording}{recording, }
      \opt{radio}{radio, }
      bookmarks                         & N/A\\
    playlist catalog directory & /path/to/dir & N/A\\
    \nopt{wheel_acceleration}{
      list\_accel\_start\_delay & 0 to 10  & ms\\
      list\_accel\_wait        & 1 to 10  & s\\
    }
%
    \opt{swcodec}{
      replaygain type
                    & track, album, track shuffle, off
                                        & N/A\\
      replaygain noclip
                    & on, off           & N/A\\
      replaygain preamp
                    & -120 to 120       & 0.1~dB\\
%
      \opt{crossfade}{
      crossfade     & off, auto track change, man track skip, shuffle,
                    shuffle or man track skip, always
                                        & N/A\\
      crossfade fade in delay
                    & 0 to 7            & s\\
      crossfade fade out delay
                    & 0 to 7            & s\\
      crossfade fade in duration
                    & 0 to 15           & s\\
      crossfade fade out duration
                    & 0 to 15           & s\\
      crossfade fade out mode
                    & crossfade, mix    & N/A\\
      }
%
      crossfeed     & on, off           & N/A\\
      crossfeed direct gain
                    & 0 to 60           & 0.1~dB\\
      crossfeed cross gain
                    & 30 to 120         & 0.1~dB\\
      crossfeed hf attenuation
                    & 60 to 240         & 0.1~dB\\
      crossfeed hf cutoff
                    & 500 to 2000       & Hz\\
%
      eq enabled    & on, off           & N/A\\
      eq precut     & 0 to 240          & 0.1~dB\\
      eq low shelf filter & cutoff (in Hz), q (0 to 64), gain (-240 to 240)\\
      eq peak filter 1 & cutoff (in Hz), q (0 to 64), gain (-240 to 240)\\
      eq peak filter 2 & cutoff (in Hz), q (0 to 64), gain (-240 to 240)\\
      eq peak filter 3 & cutoff (in Hz), q (0 to 64), gain (-240 to 240)\\
      eq peak filter 4 & cutoff (in Hz), q (0 to 64), gain (-240 to 240)\\
      eq peak filter 5 & cutoff (in Hz), q (0 to 64), gain (-240 to 240)\\
      eq peak filter 6 & cutoff (in Hz), q (0 to 64), gain (-240 to 240)\\
      eq peak filter 7 & cutoff (in Hz), q (0 to 64), gain (-240 to 240)\\
      eq peak filter 8 & cutoff (in Hz), q (0 to 64), gain (-240 to 240)\\
      eq high shelf filter & cutoff (in Hz), q (0 to 64), gain (-240 to 240 (0.1~dB))\\
%
      dithering enabled & on, off       & N/A\\
%
      timestretch enabled & on, off     & N/A\\
%
      compressor threshold      & 0 to -24      & -3~dB\\
      compressor makeup gain    & off, auto     & N/A\\
      compressor ratio          & 2:1, 4:1, 6:1, 10:1, limit
                                                & N/A\\
      compressor knee           & hard knee, soft knee
                                                & N/A\\
      compressor release time   & 100 to 1000   & 100~ms\\
%
      beep          & off, weak, moderate, strong & N/A\\
      keyclick      & off, weak, moderate, strong & N/A\\
      keyclick repeats & on, off        & N/A\\
      dircache      & on, off           & N/A\\
      tagcache\_ram & on, off           & N/A\\
    }%

    \opt{touchpad}{
      \opt{GIGABEAT_PAD}{
        touchpad sensitivity  & normal, high & N/A\\
      }
      \opt{SANSA_FUZEPLUS_PAD}{
        touchpad sensitivity  & -25 to 25 & N/A\\
        touchpad deadzone     & 0 to 100 & N/A\\
      }
    }%

    \opt{masf}{
      loudness      & 0 to 17           & N/A\\
      superbass     & on, off           & N/A\\
      auto volume   & off, 20ms, 2s, 4s, 8s
                                        & N/A\\
      mdb enable    & on,off            & N/A\\
      mdb strength  & 0 to 127          & dB\\
      mdb harmonics & 0 to 100          & \%\\
      mdb center    & 20 to 300         & Hz\\
      mdb shape     & 50 to 300         & Hz\\
    }%

    \opt{lcd_bitmap}{
      peak meter release
                    & 1 to 126          & ?\\
       peak meter hold
                    & off, 200ms, 300ms, 500ms, 1, 2, 3, 4, 5, 6, 7, 8, 9, 10,
                      15, 20, 30, 1min  & N/A \\
       peak meter clip hold
                    & on, 1, 2, 3, 4, 5, 6, 7, 8, 9, 10, 15, 20, 25, 30, 45,
                      60, 90, 2min, 3min, 5min, 10min, 20min, 45min, 90min
                                        & N/A \\
      peak meter busy & on, off         & N/A\\
      peak meter dbfs & on, off         & on:~dbfs, off:~linear\\
      peak meter min  & 0 to 89 (dB) or 0 to 100 (\%)
                                        & dB or \%\\
      peak meter max  & 0 to 89 /(dB) or 0 to 100 (\%)
                                        & dB or \%\\
      statusbar     & off, top, bottom  & N/A\\
      \opt{remote}{
        remote statusbar & off, top, bottom & N/A\\
      }
      scrollbar     & off, left, right  & N/A\\
      scrollbar width & 3 to LCD width / 10 (\fixme{devise a way
                    to get ranges from config-*.h})& pixels\\
      volume display
                    & graphic, numeric  & N/A\\
      battery display
                    & graphic, numeric  & N/A\\
      font          & /path/filename.fnt & N/A\\
      kbd           & /path/filename.kbd & N/A\\
      \opt{lcd_invert}{
        invert        & on, off           & N/A\\
      }
      \opt{lcd_flip}{
        flip display  & on, off           & N/A\\
      }
      selector type   & pointer, bar (inverse)
        \opt{lcd_color}{, bar (color), bar (gradient)} & N/A\\
      show icons    & on, off           & N/A\\
      iconset       & /path/filename.bmp & N/A\\
      viewers iconset & /path/filename.bmp & N/A\\
    }%

    \opt{swcodec}{% This doesn't depend on swcodec but using a \nopt here
                  % causes ondiosp not to build for mysterious reasons.
      backdrop      & /path/filename.bmp    & N/A\\
    }%

    \opt{lcd_color}{
      foreground colour & 000000 to FFFFFF   & RRGGBB\\
      background colour & 000000 to FFFFFF   & RRGGBB\\
      line selector start colour & 000000 to FFFFFF  & RRGGBB\\
      line selector end colour   & 000000 to FFFFFF  & RRGGBB\\
      line selector text colour  & 000000 to FFFFFF  & RRGGBB\\
      filetype colours & /path/filename.colours & N/A\\
    }

    \opt{HAVE_REMOTE_LCD}{
      rwps      & /path/filename.rwps   & N/A\\
      remote contrast
                & 5 to 63               & N/A\\
      remote invert
                & on, off               & N/A\\
      remote flip display
                & on, off               & N/A\\
      remote backlight timeout
                & off, on, 1, 2, 3, 4, 5, 6, 7, 8, 9, 10, 15, 20, 25,
                  30, 45, 60, 90        & s\\
      remote backlight timeout plugged
                & off, on, 1, 2, 3, 4, 5, 6, 7, 8, 9, 10, 15, 20, 25,
                  30, 45, 60, 90        & s\\
      remote caption backlight
                & on, off               & N/A\\
      remote scroll speed
                & 0 to 15               & N/A\\
      remote scroll step
                & 1 to 160              & N/A\\
      remote scroll delay
                & 0 to 2500             & ms\\ 
      remote bidir limit
                & 0 to 200              & N/A\\
      backlight filters first remote keypress
                & on, off               & N/A\\
      remote iconset & /path/filename.bmp & N/A\\
      remote viewers iconset & /path/filename.bmp & N/A\\
      \opt{iriverh100,iriverh300}{
        remote reduce ticking
                & on, off               & N/A\\
      }%
    }
    \opt{rtc}{
      time format & 12hour, 24hour      & N/A\\
    }%
    \opt{recording}{
     rec quality & 0 to 7               & 0: small size, 7: high quality\\
     rec frequency
                & 48, 44, 32, 24, 22, 16 & kHz\\
     rec source & mic, line, spdif      & N/A\\
     rec channels & mono, stereo        & N/A\\
     rec mic gain & 0 to 15             & N/A\\
     rec left gain & 0 to 15            & N/A\\
     rec right gain
                & 0 to 15               & N/A\\
     editable recordings
                & off,on                & N/A\\
     rec timesplit
                & off, 0:05, 0:10, 0:15, 0:30, 1:00, 2:00, 4:00, 6:00,
                  8:00, 16:00, 24:00    & h:mm\\
     pre-recording time
                & off, 1 to 30          & s\\
     rec path & /path/to/dir            & N/A\\
    }%
    \opt{spdif_power}{
      spdif enable & off, on            & N/A\\
    }%
    \opt{radio}{
      force fm mono
                & off, on               & N/A\\
    }%
    \opt{player}{
      jump scroll
                & 0 to 5                & N/A\\
      jump scroll delay
                & 0 to 250              & 0.01~s\\
    }%

    \bottomrule
  \end{longtable}
\end{center}


\input{appendix/menu_structure.tex}

\chapter{User feedback}\label{sec:feedback}
\section{Bug reports}
If you experience inappropriate performance from any supported feature,
please file a bug report on our web page. Do not report missing
features as bugs, instead file them as feature ideas (see below).

For open bug reports refer to
\url{http://www.rockbox.org/tracker/index.php?type=2}

\subsection{Rules for submitting new bug reports}

\begin{enumerate}
\item  Check that the bug has not already been reported
\item  Always include the following information in your bug report:

\begin{itemize}
\item  Which exact \dap{} you have.
\item  Which exact Rockbox version you are using
(Menu $\rightarrow$ System $\rightarrow$ Rockbox Info $\rightarrow$ Version)
\item  A step{}-by{}-step description of what you did and what happened
\item  Whether the problem is repeatable or a one{}-time occurrence
\item  All relevant data regarding the problem, such as playlists, MP3
files etc. (IMPORTANT!)
\end{itemize}
\end{enumerate}

\section{Feature ideas}
To suggest an idea for a feature or to read those made by others, see
\url{http://forums.rockbox.org/index.php?board=49.0}.  Please keep in
mind that this forum is for the discussion of feature ideas - they are not
 requests and there is no guarantee they will be acted upon.

\subsection{Rules for submitting a new feature idea}

\begin{enumerate}
\item Check that the feature has not already been suggested. 
  Duplicates are really boring!
\item Check that the feature has not already been implemented. 
  Download the latest current/daily build and/or search the mail list archive.
\item Check that the feature is possible to implement (see \reference{ref:NODO}).
\end{enumerate}

\subsection{\label{ref:NODO}Features we will not implement}
This is a list of Feature Requests we get repeatedly that we simply
cannot do. View it as the opposite of a TODO!

\begin{itemize}
\opt{archos}{
\item Record to WAV (uncompressed) or MP3pro format.\\
The recording hardware (the MAS) does not allow us to do this
\item Crossfade between tracks.\\
  Crossfading would require two mp3 decoders, and we only have one. 
  This is not possible.
\item Support MP3pro, WMA or other sound format playback.\\
  The mp3{}-decoding hardware can only play MP3. We cannot make it play other 
  sound formats.
\item  Converting OGG $\rightarrow$ MP3.\\
  The mp3{}-decoding hardware cannot decode OGG. It can be reprogrammed, but 
  there is too little memory for OGG and we have no documentation on how to 
  program the MAS' DSP. Doing the conversion with the CPU is impossible, since 
  a 12~MHz SH1 is far too slow for this daunting task.
\item Archos Multimedia support.\\
  The Archos Multimedia is a completely different beast. It is an entirely 
  different architecture, different CPU and upgrading the software is done 
  a completely different way. We do not wish to venture into this.  Others 
  may do so. We will not.
\item Multi{}-band (or graphic) equaliser.\\
  We cannot access information for that kind of visualisation from the MP3 
  decoding hardware.
\item CBR recording.\\
  The MP3 encoding hardware does not allow this.
\item Change tempo of a song without changing pitch.\\
  The MP3 decoding hardware does not allow this.
\item Graphic frequency (spectrum analyser).\\
  We cannot access the audio waveform from the MP3 decoder so we cannot analyse 
  it. Even if we had access to it, the CPU would probably be too slow to 
  perform the analysis anyway.
\item Cool sound effects.\\
  Adding new sound effects requires reprogramming the MAS chip, and we cannot 
  do that. The MAS chip is programmable, but we have no access to the chip 
  documentation.
}
\nopt{iriverh300,iaudiox5}{
\item Interfacing with other USB devices (like cameras) or 2 player games over USB.\\
  The USB system demands that there is a master that talks to a slave. The
  \dap{} can only serve as a slave, as most other USB devices such as
  cameras can. Thus, without a master no communication between the slaves
  can take place. If that is not enough, we have no way of actually
  controlling the communication performed over USB since the USB circuit
  in the \dap{} is strictly made for disk{}-access and does not allow us
  to play with it the way we'd need for any good communication to work.
}
\item Support other file systems than FAT32 (like NTFS or ext2 etc.).\\
  No.
  \opt{archos}{Rockbox needs to support FAT32 since it can only start off a FAT32
  partition (since that is the only way the ROM can load it), and adding}%
  support for more file systems will just take away valuable ram for
  unnecessary features. You can partition your \dap{} fine, just make sure
  the first one is FAT32 and then make the other ones whatever file system
  you want. Just do not expect Rockbox to understand them.
\item Add scandisk{}-like features.\\
  It would be a very slow operation that would drain the batteries and
  take a lot of useful ram for something that is much better and faster
  done when connected to a host computer.
\item Alphabetical list skipping.\\
  Skipping around the lists by jumping letters (i.e skip all C's and go
  straight to the first D). This isn't feasible with the current list
  implementation, if you really want this you can get similar effects using
  the database (see \reference{ref:database}).
\item Add support for non standard tag formats.\\
APE tags in MP3 files has been rejected a few times already. Its not something we want.
\item Implementing the ability to playback DRM files.\\
  Firstly, this would be extremely difficult to implement legally - Rockbox
  is not legal entity as such, and therefore is unable to enter into license
  agreements with providers of DRM technology.
  Secondly, Rockbox is open source, which would mean that any DRM technology we
  incorporated into our codebase would suddenly become visible to the whole world,
  completely defeating its purpose. Remember, DRM achieves part of its security
  through obscurity, and publishing the keys necessary to decrypt DRM'd 
  media would essentially render it useless.
\end{itemize}

\chapter{Credits}
People that have contributed to the project, one way or another. Friends!
%
\begin{multicols}{2}
\noindent\caps{\small{\input{CREDITS.tex}}}
\end{multicols}

\chapter{Licenses}

\section{GNU Free Documentation License}
\input{appendix/fdl.tex}
\newpage
\section{The GNU General Public License}
\input{appendix/gpl-2.0.tex}
