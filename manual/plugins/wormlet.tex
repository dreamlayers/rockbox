\subsection{Wormlet}
\screenshot{plugins/images/ss-wormlet}{Wormlet game}{img:wormlet}
Wormlet is a \opt{RECORDER_PAD}{multi{}-user }multi{}-worm game on a multi{}-threaded
multi{}-functional Rockbox console. You navigate a hungry little worm.
Help your worm to find food and to avoid poisoned argh{}-tiles. The
goal is to turn your tiny worm into a big worm for as long as possible.

\opt{RECORDER_PAD}{
For 2{}-player games a remote control is not necessary but recommended.
If you try to hold the \dap\ in the four hands of two players
you'll find out why. Games with three players are only
possible using a remote control.\\}


%The following table is only for the recorder version of the game, since the
%other versions do not support either multi player or different control modes.
%It is however prepared for the other targets should they ever support these
%features. Also some other parts of the text is "opted" out for these targets.

{\bfseries
Game controls:}

\opt{RECORDER_PAD}{
\renewcommand{\arraystretch}{1.8}
\begin{rbtabular}{\textwidth}{c X p{3cm} p{3cm} p{3cm}}%
{\textbf{Players} & \textbf{Modes} & \textbf{Player 1} & \textbf{Player 2}
                        & \textbf{Player 3}}{}{}
%
0 & Out of control & \multicolumn{3}{p{9cm}}{With no player taking part in the
    game all worms are out of control and steered by artificial stupidity.}\\
%
\multirow{2}{*}{1} & 2 key control & on \dap\ \ButtonLeft: turn left
                        \ButtonRight: turn right & {}- & {}-\\
                    & 4 key control & on \dap\
                        \ButtonLeft: turn  left
                        \ButtonUp: turn up
                        \ButtonRight: turn right
                        \ButtonDown: turn down & {}- & {}- \\
%
\multirow{2}{*}{2} & Remote control & on \dap\ \ButtonLeft: turn left
                        \ButtonRight: turn right & on remote control VOL DOWN:
                        turn left VOL UP: turn right & {}- \\
                    & No remote control & on \dap\ \ButtonLeft: turn left
                        \ButtonRight: turn right & on \dap\ \ButtonFTwo: turn
                        left \ButtonFThree: turn right & {}- \\
3 & Remote control & on \dap\ \ButtonLeft: turn left \ButtonRight: turn right
                        & on remote control VOL DOWN: turn left VOL UP: turn
                        right & on \dap\ \ButtonFTwo: turn left \ButtonFThree:
                        turn right \\
\end{rbtabular}
\renewcommand{\arraystretch}{1.0}
}

\nopt{RECORDER_PAD}{
    \begin{btnmap}
        \nopt{MPIO_HD200_PAD,MPIO_HD300_PAD,touchscreen}{\ButtonLeft}
        \opt{MPIO_HD200_PAD}{\ButtonVolDown}
        \opt{MPIO_HD300_PAD}{\ButtonRew}
        \opt{touchscreen}{\TouchMidLeft}
            &
        \opt{HAVEREMOTEKEYMAP}{
            &}
        Turn left
        \\
        
        \nopt{MPIO_HD200_PAD,MPIO_HD300_PAD,touchscreen}{\ButtonRight}
        \opt{MPIO_HD200_PAD}{\ButtonVolUp}
        \opt{MPIO_HD300_PAD}{\ButtonFF}
        \opt{touchscreen}{\TouchMidRight}
            &
        \opt{HAVEREMOTEKEYMAP}{
            &}
        Turn right
        \\
        
        \nopt{IPOD_3G_PAD,IPOD_4G_PAD,IRIVER_H10_PAD,MPIO_HD200_PAD%
            ,MPIO_HD300_PAD,touchscreen}{\ButtonUp}
        \opt{IPOD_3G_PAD,IPOD_4G_PAD}{\ButtonMenu}
        \opt{IRIVER_H10_PAD}{\ButtonScrollUp}
        \opt{MPIO_HD200_PAD}{\ButtonRew}
        \opt{MPIO_HD300_PAD}{\ButtonScrollUp}
        \opt{touchscreen}{\TouchTopMiddle}
            &
        \opt{HAVEREMOTEKEYMAP}{
            &}
        Turn Up
        \\
        
        \nopt{IPOD_3G_PAD,IPOD_4G_PAD,IRIVER_H10_PAD,MPIO_HD200_PAD%
            ,MPIO_HD300_PAD,touchscreen}{\ButtonDown}
        \opt{IPOD_3G_PAD,IPOD_4G_PAD}{\ButtonPlay}
        \opt{IRIVER_H10_PAD}{\ButtonScrollDown}
        \opt{MPIO_HD200_PAD}{\ButtonFF}
        \opt{MPIO_HD300_PAD}{\ButtonScrollDown}
        \opt{touchscreen}{\TouchBottomMiddle}
            &
        \opt{HAVEREMOTEKEYMAP}{
            &}
        Turn Down
        \\
    \end{btnmap}
}

\subsubsection{The game}
Use the control keys of your worm to navigate around obstacles and find
food. Worms do not stop moving except when dead. Dead worms are no fun.
Be careful as your worm will try to eat anything that you steer it
across. It won't distinguish whether it is edible or not.

\begin{description}
\item[Food.]
The small square hollow pieces are food. Move the worm over a food tile
to eat it. After eating the worm grows. Each time a piece of food has
been eaten a new piece of food will pop up somewhere. Unfortunately for
each new piece of food that appears two new ``argh'' pieces will
appear, too.
\item[Argh.]
An ``argh'' is a black square poisoned piece {}- slightly bigger than
food {}- that makes a worm say ``Argh!'' when
run into.  A worm that eats an ``argh'' is dead. Thus eating an
``argh'' must be avoided under any circumstances. ``Arghs'' have the
annoying tendency to accumulate. 
\item[Worms.]
Thou shall not eat worms. Neither other worms nor thyself. Eating worms
is blasphemous cannibalism, not healthy and causes instant
death. And it doesn't help anyway: the other worm
isn't hurt by the bite. It will go on creeping happily
and eat all the food you left on the table. 
\item[Walls.]
Don't crash into the walls. Walls are not edible.
Crashing a worm against a wall causes it a headache it
doesn't survive. 
\item[Game over.]
The game is over when all worms are dead. The longest worm wins the
game. 
\item [Pause the game.]
Press
\opt{RECORDER_PAD,IAUDIO_X5_PAD,PBELL_VIBE500_PAD,MPIO_HD200_PAD%
    ,MPIO_HD300_PAD,SAMSUNG_YH92X_PAD,SAMSUNG_YH820_PAD}{\ButtonPlay}%
\opt{ONDIO_PAD}{\ButtonMenu}%
\opt{IRIVER_H100_PAD,IRIVER_H300_PAD,IPOD_4G_PAD,IPOD_3G_PAD,SANSA_E200_PAD,SANSA_FUZE_PAD%
  ,GIGABEAT_PAD,GIGABEAT_S_PAD}{\ButtonSelect}
\opt{COWON_D2_PAD}{\TouchCenter}
to pause the game. Press it again to resume the game.

\item[Stop the game.]
There are two ways to stop a running game.

\begin{itemize}
\item If you want to quit Wormlet entirely simply hit
\opt{RECORDER_PAD,ONDIO_PAD,IRIVER_H100_PAD,IRIVER_H300_PAD}{\ButtonOff}%
\opt{IPOD_4G_PAD,IPOD_3G_PAD}{\ButtonMenu+\ButtonSelect}%
\opt{IAUDIO_X5_PAD,SANSA_E200_PAD,GIGABEAT_PAD}{\ButtonPower}%
\opt{SANSA_FUZE_PAD}{Long \ButtonHome}%
\opt{PBELL_VIBE500_PAD}{\ButtonRec}%
\opt{SAMSUNG_YH92X_PAD,SAMSUNG_YH820_PAD}{\ButtonRew}%
\opt{MPIO_HD200_PAD}{\ButtonRec + \ButtonPlay}%
\opt{MPIO_HD300_PAD}{Long \ButtonMenu}%
\opt{GIGABEAT_S_PAD}{\ButtonBack}.
The game will stop immediately and you will return to the game menu. 
\item If you want to stop the game and still see the screen hit 
\opt{RECORDER_PAD,IRIVER_H100_PAD,IRIVER_H300_PAD}{\ButtonOn}%
\opt{ONDIO_PAD}{\ButtonOff+\ButtonMenu}%
\opt{IPOD_4G_PAD,IPOD_3G_PAD}{\ButtonSelect+\ButtonPlay}%
\opt{IAUDIO_X5_PAD,SANSA_E200_PAD}{\ButtonRec}%
\opt{SANSA_FUZE_PAD}{\ButtonSelect+\ButtonUp}%
\opt{GIGABEAT_PAD}{\ButtonA}%
\opt{PBELL_VIBE500_PAD}{\ButtonCancel}%
\opt{SAMSUNG_YH92X_PAD,SAMSUNG_YH820_PAD}{\ButtonFF}%
\opt{MPIO_HD200_PAD}{Long \ButtonFunc}%
\opt{MPIO_HD300_PAD}{Long \ButtonPlay}%
\opt{GIGABEAT_S_PAD}{\ButtonMenu}.
This freezes the game. If you hit
\opt{RECORDER_PAD,IRIVER_H100_PAD,IRIVER_H300_PAD}{\ButtonOn}%
\opt{ONDIO_PAD}{\ButtonOff+\ButtonMenu}%
\opt{IPOD_4G_PAD,IPOD_3G_PAD}{\ButtonSelect+\ButtonPlay}%
\opt{IAUDIO_X5_PAD,SANSA_E200_PAD}{\ButtonRec}%
\opt{SANSA_FUZE_PAD}{\ButtonSelect+\ButtonUp}%
\opt{GIGABEAT_PAD}{\ButtonA}%
\opt{PBELL_VIBE500_PAD}{\ButtonCancel}%
\opt{SAMSUNG_YH92X_PAD,SAMSUNG_YH820_PAD}{\ButtonFF}%
\opt{MPIO_HD200_PAD}{Long \ButtonFunc}%
\opt{MPIO_HD300_PAD}{Long \ButtonPlay}%
\opt{GIGABEAT_S_PAD}{\ButtonMenu}
button again a new game starts with the same configuration. To return to the
games menu you can hit
\opt{RECORDER_PAD,ONDIO_PAD,IRIVER_H100_PAD,IRIVER_H300_PAD}{\ButtonOff}%
\opt{IPOD_4G_PAD,IPOD_3G_PAD}{\ButtonMenu+\ButtonSelect}%
\opt{IAUDIO_X5_PAD,SANSA_E200_PAD,GIGABEAT_PAD}{\ButtonPower}%
\opt{SANSA_FUZE_PAD}{Long \ButtonHome}
\opt{PBELL_VIBE500_PAD}{\ButtonRec}%
\opt{SAMSUNG_YH92X_PAD,SAMSUNG_YH820_PAD}{\ButtonRew}%
\opt{MPIO_HD200_PAD}{\ButtonRec + \ButtonPlay}%
\opt{MPIO_HD300_PAD}{Long \ButtonMenu}%
\opt{GIGABEAT_S_PAD}{\ButtonBack}. A stopped game can not be resumed. 
\end{itemize}
\end{description}

\subsubsection{The scoreboard}
On the right side of the game field is the score board. For each worm it
displays its status and its length. The top most entry displays the
state of worm 1, the second worm 2 and the third worm 3. When a worm
dies its entry on the score board turns black.

\begin{description}
\item[Len:]
Here the current length of the worm is displayed. When a worm is eating
food it grows by one pixel for each step it moves. 

\item[Hungry:]
That's the normal state of a worm. Worms are always
hungry and want to eat. It is good to have a hungry
worm since it means that your worm is alive. But it is
better to get your worm growing. 

\item[Growing:]
When a worm has eaten a piece of food it starts growing. For each step
it moves over food it can grow by one pixel. One piece of food lasts
for 7 steps. After your worm has moved 7 steps the food is used up. If
another piece of food is eaten while growing it will increase the size
of the worm for another 7 steps. 

\item[Crashed:]
This indicates that a worm has crashed against a wall.

\item[Argh:]
If the score board entry displays ``Argh!'' it
means the worm is dead because it tried to eat an ``argh''. Until we
can make the worm say ``Argh!'' it is your job to say ``Argh!'' aloud.

\item[Wormed:]
The worm tried to eat another worm or even itself.
That's why it is dead now.  Making traps for other players with a worm
is a good way to get them out of the game.
\end{description}

\subsubsection{Hints}

\begin{itemize}

\item Initially you will be busy with controlling your worm. Try to
avoid other worms and crawl far away from them. Wait until they curl up
themselves and collect the food afterwards. Don't worry if the other
worms grow longer than yours {}- you can catch up after they've died. 

\item When you are more experienced watch the tactics of other worms.
Those worms controlled by artificial stupidity head straight for the
nearest piece of food. Let the other worm have its next piece of food
and head for the food it would probably want next. Try to put yourself
between the opponent and that food. From now on you can `control' the
other worm by blocking it. You could trap it by making a 1 pixel wide
U{}-turn. You also could move from food to food and make sure you keep
between your opponent and the food. So you can always reach it before
your opponent. 

\opt{RECORDER_PAD}{
\item While playing the game the \dap\ can still play music. For
single player game use any music you like. For berserk games with 2 players use
hard rock and for 3 player games use heavy metal or X{}-Phobie
(\url{http://www.x-phobie.de/}).
Play fair and don't kick your opponent in the toe or
poke him in the eye. That would be bad manners.
}
\end{itemize}
